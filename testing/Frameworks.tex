%        File: Frameworks.tex
%     Created: Sat Mar 26 09:00 pm 2016 G
% Last Change: Sat Mar 26 09:00 pm 2016 G
%
\documentclass[a4paper, titlepage]{article}

% We include the color package in the below included
% file which sets up some custom colors
\usepackage{xcolor}
%Colour defines, see latexcolor.com for many many more
\definecolor{cornflowerblue}{rgb}{0.39,0.58,0.93}%#6495ED
\definecolor{cosmiclatte}{rgb}{1.0,0.97,0.91}%#FFFE87
\definecolor{cobalt}{rgb}{0.0,0.28,0.67}%#FFFE87

\newcommand{\ac}[2]{
  \textcolor{cobalt}{\texttt{#1 [\textcolor{red}{#2}]}}
}


\usepackage{fontspec}
\usepackage[urlcolor=blue, colorlinks=false]{hyperref}

\usepackage[breakable]{tcolorbox}
\tcbuselibrary{minted}
\tcbuselibrary{breakable}
% Box defines, we are using package tcolorbox, this is not added here
% but in the main file. This file should then be inputted directly
% below.

\newtcolorbox{spec}[1]{before=\par\smallskip\centering, after=\par,
width=#1\linewidth, colback=cosmiclatte, colframe=cobalt, halign=center}

\newtcolorbox[auto counter, number within=section]{tspec}[2]{before=\par\smallskip\centering, after=\par,
width=#1\linewidth, colback=cosmiclatte, colframe=cobalt,
fonttitle=\bfseries, title=Example~\thetcbcounter: #2, halign=center,
label={Ex:#2}}

\newtcblisting[auto counter, number within=section]{listbox}[2]{
width=#1\linewidth, colback=cosmiclatte, colframe=cobalt,
fonttitle=\bfseries, title=Listing~\thetcbcounter: #2, halign=center,
label={Cd:#2}, breakable, before=\par\smallskip\centering, after=\par,
listing only, minted language=objc, minted options={breaklines=true,
breakanywhere=true, fontsize=\small}}


%\usepackage{minted}

\setmainfont{Open Sans}
\setmonofont{Inconsolata}

\setlength{\parindent}{0pt}
\setlength{\parskip}{1em}


%\newcommand{\ac}[2]{\textcolor{cobalt}{\texttt{#1 [\textcolor{red}{#2}]}}}

\title{An Investigation into frameworks for Test Driven
Development on the iOS platform with further
analysis of Business Driven Development as it relates to test naming
conventions.}

\author{Tim Storey \hfill for \hfill ICSA Software Ltd.
        \\\texttt{tim.storey@boardpad.com}}

\date{June, 2016}

\begin{document}


\maketitle

\begin{abstract}
  The document presents an analysis of various testing frameworks
  and the merits of using matchers over the native assertions of
  XCTest. The document will further consider the  BDD syntax available with the
  libraries and its merits in terms of naming as it relates to tests as code documentation
  and ease of description of both business logic and unit test as well
  as its use in ticket creation and deliverable estimations.
  \end{abstract}

%{\hypersetup{linkcolor=black} \tableofcontents } 
%\tableofcontents
\newpage

\section{Introduction}
  \label{sec:Intro}
  Unit testing is fairly new in regards to iOS, it was first added to
  Xcode5 as a native library in the form of XCTest although it has been available
  as the SenTestingKit for OSx since Xcode 2. As such it has not become such an
  integral part of a development strategy as it has in other
  platforms particularly in the world of Java based server development.

  A discussion of the merits or otherwise of TDD is beyond the scope of
  this document, here we assume that it \textbf{IS} desirable to write
  code in a test first manner and as such want to help this aim with the best
  tool kit for the job.
  
  There has also been move toward allowing for a 
  BDD\footnote{\url{https://en.wikipedia.org/wiki/Behavior-driven_development}} 
  like syntax whereby a system is
  described in a language more suited to the business needs itself, this
  language attempts to match the actual behavior and expectations of
  the system.\\For example we might define a requirement as a user
  story in the following manner capturing \textit{role},
  \textit{feature} and \textit{benefit};

  \begin{spec}{0.5}
    \ac{As a }{ROLE}\\
    \ac{I want }{FEATURE}\\
    \ac{So that }{VALUE/OUTCOME}
  \end{spec}

  Then we might decompose each part of the user story in a
  manner that explains how the user expects the system to behave, i.e. we
  would describe a scenario in terms of acceptance criteria;

  \begin{spec}{0.5}
    \ac{Given:}{some CONTEXT}\\
    \ac{When:}{some EVENT}\\
    \ac{Then:}{some ACTION}
  \end{spec}

  It is becoming common to specify behaviours this way as it allows
  scenarios to be described in manner which allows developers to build
  up code that matches the requirements easily without wasted effort and
  it also provides acceptance criteria. It is however worth noting that
  this will never be perfect and a degree of pragmatism is as always
  needed.
  
  XCTest is Apple's version of OCUnit/SenTestingKit and is quite an
  old library,
  as such whilst OCUnit works well it was never really written with more
  modern testing practices in mind, by which I mean the use of a BDD like syntax. It does
  not have any support for matchers (although these can be added) or context based tests furthermore
  asynchronous testing is only a recent addition. These 
  limitations that are fine when testing in a \textit{Test Last}
  situation (i.e. Unit Testing) and also work fine for simple
  TDD are less useful when we wish to consider our code as not only
  fulfilling requirements but also acting as documentation for later
  developers (ourselves included!)\\

\section{BDD syntax}
  BDD\footnote{\url{https://en.wikipedia.org/wiki/Behavior-driven_development}}
  is a relatively new (it was formally defined in 2009) process that is in
  essence an extension of TDD methodologies. As briefly explained in the
  section~\nameref{sec:Intro} it is a means of extending TDD and
  providing a syntax that closely matches Agile's (and to some extent UML's) User
  Stories which further creates acceptance testing criteria. 
  It has a simple syntax and
  grammar that allows the capture and expression of concepts
  understandable to the whole business.

  Essentially we can view this as an attempt (and rather a successful
  one) at bridging the languages that exists through varying levels of
  a business, from Product Owners to the Business and from Developers to
  Business Analysts/Project Manager. 
  
  To explain this a little more most commonly a business will tend to be split
  into three language \textit{levels}, by which I mean three different approaches
  to the common goal and thus three entirely different manners of
  expressing the goal.

  The \textit{levels} might be considered to be something like
  
  \begin{enumerate}
    \item{Business/Product Owner}
    \item{Business Analyst/Product Manager}
    \item{Developers}
  \end{enumerate}
  
  Using BDD we try to create a bridge between all the \textit{levels} and thus
  work toward ensuring that the team communicates each part's needs
  effectively to the other parts of the business. 

  As such we end up with a series of logical \textit{contexts} that define
  units of work, both in terms of business logic (that managers
  understand) and functional logic (that developers understand).

  There is no formal strict specification of how this should be done,
  the given team must reach its own agreement of the best means of
  describing the various levels of story.

  For example let us consider the following requirement that is actually
  in an existing ticket. 

  \subsection{A worked Example}

  The business wants an icon to be displayed on a cell if the document can be backed
  up using CrossPlatformSync, a new feature just added to the product. 
  This is a simple example but shows the general idea.

  We might capture this as a User Story in the following manner
  \begin{tspec}{1.0}{User Story}
    \centering
      \ac{As a}{USER}\\
      \ac{I want}{To see an icon displayed on a document icon}\\
      \ac{So that}{I know I can backup my document across platforms}
  \end{tspec}

  We can break this down further by driving out the \textbf{Acceptance
  Criteria}.
  
  These allow us to know when the task can be considered
  complete, these criteria are sometimes referred to as
  \textbf{scenarios} or sometimes \textbf{specifications} and this
  naming tends to be determined by the tooling used to write the tests.
  For our purposes we will refer to these as specifications or specs.
  
  So we end up with a spec that might look something like this.

  \begin{tspec}{1}{Cross platform sync icon display}
      \ac{Given}{I am working online}\\
    \ac{And}{I have logged in successfully}\\
    \ac{And}{I have documents that I have permissions to view}\\
    \ac{And}{I have downloaded these documents}\\
    \ac{Then}{I want to see those documents listed}\\
    \ac{And}{Cross platform sync is available}\\
    \ac{Then}{I want to see the cross platform sync icon}\\
  \end{tspec}

  As can be seen we have now managed to drive out a lot of separate
  acceptance criteria that are all involved in the original user story \hyperref[Ex:User Story]{Example 2.1 User Story}
  so we might think about creating some more user stories, perhaps
  something along the lines of

  \begin{itemize}
    \item{User authorization}
    \item{Document download}
  \end{itemize}

  These themselves may have acceptance criteria look something like

  \begin{tspec}{1}{User Authorization}
    \ac{As a}{USER}\\
    \ac{I want}{To be able to login}\\
    \ac{So that}{I can use the application}
  \end{tspec}

  \begin{tspec}{1}{Document download}
    \ac{As a}{USER}\\
    \ac{I want}{To be able see a list of my documents}\\
    \ac{So that}{I can use them}\\
  \end{tspec}

  All of the derived acceptance criteria can be considered to be discrete
  logic areas that will require development work, and as such become 
  our \textbf{requirements} or in terms of a testing framework that supports BDD 
  \textbf{contexts}.
  
  We will be developing the code to reach these criteria in a test
  first manner and we can build up the tests and thus the code wrapped
  in one of the contexts that we have just defined.

  These requirements further describe the integration tests
  that QA will need to do, but we will come back to this in \hyperref[sec:integration tests]{Functional/Integration tests}
  
  This example is somewhat contrived (though based on a ticket) and could 
  be worked further but it is not the intention to
  fully explore BDD here merely to show its utility.

\section{Specs as tickets and estimation}
  As we can see the BDD syntax drives out the criteria we need to meet
  for the story to be accepted and as such we are now in a better
  position to divide the needed development work up into distinct units which can be
  captured as tickets.

  Considering the requirements for \textbf{Cross Platform Sync icon display}
  and assuming we have now refined the specs to the following;

  \begin{tspec}{1}{Display Cross Platform Sync Icon}
    \ac{Given}{There are documents displayed for a meeting}\\
    \ac{And}{Cross platform sync is enabled}\\
    \ac{Then}{I want to see the cross platform sync icon displayed}
  \end{tspec}

  the specs so far are descriptive enough to capture
  the story at a fairly high level and can be easily enough
  understood across the levels of the business  to reach agreement
  between the teams but in terms of
  development we are going to need to go deeper and create finer grained
  technical specifications that achieve the goals set out in the
  requirements, really it's these technical specs
  (that we might consider a sub category of a BDD User Story) that we
  are most interested in for the purposes of this document.

  These technical specs will define the functional requirements of the code we are
  developing.

  \subsection{Developing functional requirements and tickets}
   Let's assume that most of the code for the spec is already covered i.e. documents
  are downloaded and displayed correctly we just need to write some code
  to set a \texttt{Boolean} flag that indicates that the icon needs display.

  We have previously decided on an overall architecture and know that the document
  icons are displayed as cells in a table view, this has a working data
  source etc so that takes care of the \texttt{Given} part of the
  spec. 

  Given the above assumption we could now make 2 tickets as subtasks to capture the
  deliverables that represent the story, if we had no data source nor
  table view then that of course would become another deliverable
  albeit a big one.
  
  So then these tickets are going to be 

  \begin{enumerate}
    \item{\texttt{Determine if cross platform sync is enabled}}
    \item{\texttt{Display icon if cross platform is enabled}}
  \end{enumerate}
  
  We now have small estimate able tickets that capture the information the
  business need in terms they understand but the information we now need
  as developers tasked with delivering the story is of course of a
  different nature.  

\section{BDD Syntax in Test Driven Development}

  Investigating the first ticket we discover that there are a couple of
  conditions that determine whether or not cross platform sync is
  enabled, we know that the availability of CrossPlatformSync is based
  on server version and on permissions held in the users organisation
  profile.

  The server version is available to the table view that configures the
  cell and the users permissions regarding backup are held in the
  organisation profile.

  Architecturally we decide that it makes sense to create a
  small reusable, extensible helper class that can capture the logic
  we need and return the \texttt{Boolean} value that determines whether or not to
  display the icon. 
  
  The class responsible for displaying the icon can
  then use this new helper class, and we will pass it to the class as a
  dependency thus decoupling our logic like good nerds.

  We will build this code up under test and we are therefore likely
  going to need two test cases one for the new class and one for the
  class that has a dependency on this new helper class.

  Assuming we are calling the class CrossPlatformSyncHelper we might
  end up with a test class something like the following (for the sake of
  brevity a lot of boilerplate has been removed);

  \begin{listbox}{1}{XCTest Case first parse}
      - (void)test_InitializationCalls
      {
          XCTAssertNotNil(self.crossPlatformSyncSupportHelperUnderTest, @"Did not init");
      }

      - (void)test_initialisesWithAMinimumServerVersion
      {
          self.crossPlatformSyncSupportHelperUnderTest = [CrossPlatformSyncSupportHelper supportHelperWithMinimumDocumentServerVersion:1];
          XCTAssertNotNil(self.crossPlatformSyncSupportHelperUnderTest, @"Failed to initialise class");
          XCTAssertEqual(self.crossPlatformSyncSupportHelperUnderTest.minimumDocumentServerVersion, 1, @"Failed to set server version");
      }

      - (void)test_givenADocumentServerVersionAboveZeroAndAnOrganisationProfileThatAllowsBackupThenCrossPlatformSyncReturnsTrue
      {
          [self mockBoardData];
          OrganisationProfile *organisationProfile = [OrganisationProfile insertInManagedObjectContext:self.organisationProfileManagedObjectContext];
          organisationProfile.AllowBackup = [NSNumber numberWithBool:YES];

          OCMStub([self.mockedBoardData currentOrganisationProfile]).andReturn(organisationProfile);
          XCTAssertTrue([_crossPlatformSyncSupportHelperUnderTest
          isCrossPlatformSyncEnabledForDocumentServerVersion:1
          withProfile: [self.mockedBoardData currentOrganisationProfile]], @"Does not return true for server version 1");
      }

      - (void)test_givenAnOrgProfileThatHasSyncEnabledLocalTestForSyncEnabledReturnsTrue
      {
          [self mockBoardData];
          OrganisationProfile *organisationProfile = [OrganisationProfile insertInManagedObjectContext:self.organisationProfileManagedObjectContext];
          organisationProfile.AllowBackup = [NSNumber numberWithBool:YES];

          XCTAssertTrue([_crossPlatformSyncSupportHelperUnderTest isCrossPlatformSyncEnabledForOrganisationProfile:organisationProfile], @"Does not return TRUE when valid in org profile");
      }
  \end{listbox}

  
The \mintinline{objc}{XCTest Case} above covers all the functions we
  need for our new class, we have
  
   \begin{listbox}{1}{Test names}
       test_InitializationCalls;
       test_initialisesWithAMinimumServerVersion;
       test_givenADocumentServerVersionAboveZeroAndAnOrganisationProfileThatAllowsBackupThenCrossPlatformSyncReturnsTrue;
      test_givenAnOrgProfileThatHasSyncEnabledLocalTestForSyncEnabledReturnsTrue;
  \end{listbox}

  Our tests cover instantiating the class and its
  functionality but it is hard to read the test names and it doesn't match
  any level of BDD like syntax, so although the functions are covered
  and there is nothing to stop us building up our code in a test first
  manner the final test class has little merit as documentation because
  it is so hard to read. The test names reflect their purpose but it is
  hard on the eyes.

  Testing frameworks that support BDD allow us to restructure the test
  cases into contexts via the use of a block based syntax.

  We could therefore separate the above set of tests in to 2 separate
  contexts, something like

  \begin{enumerate}
    \item{\texttt{Initialization}}
    \item{\texttt{Cross platform sync logic}}
  \end{enumerate}

  Using the Specta or Quick framework we can write a test class that
  looks like the following. 

  \begin{listbox}{1}{Context Based Version}
    describe(@"Initialization calls", ^{

      ClassUnderTest *sut;

      given(@"The pre-requisites are met", ^{

          it(@"Should initialize from its designated initializer", ^{
              sut = [ClassUnderTest alloc]init];
              expect(sut).toNot(beNil());
          });
      });

      given(@"We have initialized correctly", ^{
          
        it(@"Performs the logic we want", ^{
            expect([sut someMethod]).to.equal(someValue);
        });

      });
      
    });
  \end{listbox}

  The above class is far more readable, the use of the
  \texttt{Describe} and \texttt{Given} blocks allow us to separate the
  tests into logical areas that lead to a form of self
  documenting code.
  
  We have pretty much managed to explain how a class
  works with the \texttt{GIVEN} Something has happened, \texttt{THEN} Something else will
  happen and we \texttt{EXPECT} Something to be Something syntax.


\section{Matchers vs Assertions}

XCTest is assertion based, out of the box it provides the following
macros

\begin{listbox}{1}{XCTest Assertion Macros}
XCTFail (format);
XCTAssertNil (a1, format);
XCTAssertNotNil (a1, format);
XCTAssert (a1, format);
XCTAssertTrue (a1, format);
XCTAssertFalse (a1, format);
XCTAssertEqualObjects (a1, a2, format);
XCTAssertEquals (a1, a2, format);
XCTAssertEqualsWithAccuracy (a1, a2, accuracy, format);
XCTAssertThrows (expression, format);
XCTAssertThrowsSpecific (expression, specificException, format);
XCTAssertThrowsSpecificNamed (expression, specificException,
exceptionName, format);
XCTAssertNoThrow (expression, format);
XCTAssertNoThrowSpecific (expression, specificException, format);
XCTAssertNoThrowSpecificNamed (expression, specificExcepton,
exceptionName, format);
\end{listbox}

These are fine and work as expected but when
building up a complex set of tests that we would like to be very clear
(i.e. read a documentation as well as cover the logic)
they quickly become limited. 

The only really meaningful error we get is the \texttt{format} string 
we pass into the macro as the last parameter. Furthermore asynchronous 
testing not trivially catered for.

\begin{listbox}{1}{Asynchronous XCTest}
- (void)testDownload
    {
        XCTestExpectation *expectation =
            [self expectationWithDescription:@"High Expectations"];

        [self.pageLoader requestUrl:@"http://fooBar.com"
                  completionHandler:^(NSString *page) {
                  
            NSLog(@"The web page is %ld bytes long.", page.length);
            XCTAssert(page.length > 0);
            [expectation fulfill];
        }];
    
        [self waitForExpectationsWithTimeout:5.0 handler:^(NSError *error) {
            if (error) {
                NSLog(@"Timeout Error: %@", error);
            }
        }];
    }
\end{listbox}

Matcher frameworks act as an API to express outcomes of a code unit.

The Nimble\footnote{\url{https://github.com/Quick/Nimble}}  framework
bundled with Quick\footnote{\url{https://github.com/Quick}} provides 
us with the ability to write our tests in a manner that is clearer as the
test describes itself much in the same manner as a BDD compliant
framework allows us to use the \texttt{GIVEN, THEN, AND, SHOULD} syntax, only for
the matchers we use \texttt{EXPECT} something to \texttt{BE SOMETHING},
we have the same set of matches as with XCAsserts and more to boot.

Nimble provides the following expectations.
\begin{listbox}{1}{Nimble Matchers}
// Passes if actual is equivalent to expected:
expect(actual).to(equal(expected))

// Passes if actual is not equivalent to expected:
expect(actual).toNot(equal(expected))

// Passes if actual has the same pointer address as expected:
expect(actual).to(beIdenticalTo(expected));

// Passes if actual does not have the same pointer address as expected:
expect(actual).toNot(beIdenticalTo(expected));

expect(actual).to(beLessThan(expected));
expect(actual).to(beLessThanOrEqualTo(expected));
expect(actual).to(beGreaterThan(expected));
expect(actual).to(beGreaterThanOrEqualTo(expected));

expect(actual).to(beCloseTo(expected).within(delta));

// Passes if instance is an instance of aClass:
expect(instance).to(beAnInstanceOf(aClass));

// Passes if instance is an instance of aClass or any of its subclasses:
expect(instance).to(beAKindOf(aClass));

// Passes if actual is not nil, true, or an object with a boolean value of true:
expect(actual).to(beTruthy());

// Passes if actual is only true (not nil or an object conforming to BooleanType true):
expect(actual).to(beTrue());

// Passes if actual is nil, false, or an object with a boolean value of false:
expect(actual).to(beFalsy());

// Passes if actual is only false (not nil or an object conforming to BooleanType false):
expect(actual).to(beFalse());

// Passes if actual is nil:
expect(actual).to(beNil());

// Passes if expected is a member of actual:
expect(actual).to(contain(expected));

// Passes if actual is an empty collection (it contains no elements):
expect(actual).to(beEmpty());

// Passes if actual contains substring expected:
expect(actual).to(contain(expected));

// Passes if actual begins with substring:
expect(actual).to(beginWith(expected));

// Passes if actual ends with substring:
expect(actual).to(endWith(expected));

// Passes if actual is an empty string, "":
expect(actual).to(beEmpty());

// Passes if actual matches the regular expression defined in expected:
expect(actual).to(match(expected))

// passes if actual collection's count is equal to expected
expect(actual).to(haveCount(expected))

// passes if actual collection's count is not equal to expected
expect(actual).notTo(haveCount(expected))
\end{listbox}

and indeed more to pass tests over collections handle swift exceptions
etc. see\footnote{\url{https://github.com/Quick/Nimble#built-in-matcher-functions}} 

Essentially it can be seen that the above assertions are easier to
read at a glance than the equivalent XCTestAssertions.

Should the test fail then an error is thrown as with XCAssert however
the framework is not reliant on the passed string to report the error.

For example:

\begin{listbox}{1}{Failure example}
expect(@(1+1)).to(equal(@3));
// failed - expected to equal <3.0000>, got <2.0000>
\end{listbox}

Nimble matchers also provide simple asynchronous tests as below

\begin{listbox}{1}{Asynchronous example}
expect(foo).withTimeout(3).toEventually(contain(@"bar"));
\end{listbox}

\textbf{NB} We don't not need to add the \texttt{withTimeout()} unless we wish to wait
for more than the default 1 second.

\section{Quick framework and the Nimble matcher framework}

Quick is a testing framework written in Swift that supports a BDD syntax
and comes bundled with a matcher framework, Nimble and is interoperable
between both Swift and Objective-C 2.0. Tests can be written in both
languages.

\section{Conclusion}

  It hopefully can be seen that well written tests add clarity to a code
  base and can act as a form of documentation, this is far easier if the
  framework supports a BDD like syntax and uses an easy to read matcher
  framework.

  Tests should cover the logic of the classes under test but should
  also be expressive, and thereby easier to read quickly.

  Tests should not only over the fine grained logic of the classes under
  test but be expressive at a higher level, i.e. \texttt{GIVEN, THEN,
  AND, SHOULD, EXPECT}

  Tests should act as documentation of the code base, this is not to say
  they should be the only documentation but they should be a part of
  this.

  \texttt{Quick/Nimble} would seem to be a good choice for testing as it
  supports both Swift and Objective C and thus can be used as a means of
  starting to work in Swift by writing tests in Swift and comes with a
  comprehensive matcher framework. It also fully supports writing tests
  in Objective-C and thus presents no obstacles.
\end{document}


