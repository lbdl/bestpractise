%        File: Frameworks.tex
%     Created: Sat Mar 26 09:00 pm 2016 G
% Last Change: Sat Mar 26 09:00 pm 2016 G
%
\documentclass[a4paper, titlepage]{article}

% We include the color package in the below included
% file which sets up some custom colors
\usepackage{xcolor}
%Colour defines, see latexcolor.com for many many more
\definecolor{cornflowerblue}{rgb}{0.39,0.58,0.93}%#6495ED
\definecolor{cosmiclatte}{rgb}{1.0,0.97,0.91}%#FFFE87
\definecolor{cobalt}{rgb}{0.0,0.28,0.67}%#FFFE87

\newcommand{\ac}[2]{
  \textcolor{cobalt}{\texttt{#1 [\textcolor{red}{#2}]}}
}


\usepackage{fontspec}
\usepackage[urlcolor=blue, colorlinks=false]{hyperref}

\usepackage[breakable]{tcolorbox}
\tcbuselibrary{minted}
\tcbuselibrary{breakable}
% Box defines, we are using package tcolorbox, this is not added here
% but in the main file. This file should then be inputted directly
% below.

\newtcolorbox{spec}[1]{before=\par\smallskip\centering, after=\par,
width=#1\linewidth, colback=cosmiclatte, colframe=cobalt, halign=center}

\newtcolorbox[auto counter, number within=section]{tspec}[2]{before=\par\smallskip\centering, after=\par,
width=#1\linewidth, colback=cosmiclatte, colframe=cobalt,
fonttitle=\bfseries, title=Example~\thetcbcounter: #2, halign=center,
label={Ex:#2}}

\newtcblisting[auto counter, number within=section]{listbox}[2]{
width=#1\linewidth, colback=cosmiclatte, colframe=cobalt,
fonttitle=\bfseries, title=Listing~\thetcbcounter: #2, halign=center,
label={Cd:#2}, breakable, before=\par\smallskip\centering, after=\par,
listing only, minted language=objc, minted options={breaklines=true,
breakanywhere=true, fontsize=\small}}


%\usepackage{minted}

\setmainfont{Georgia}
\setmonofont{Inconsolata}

\setlength{\parindent}{0pt}
\setlength{\parskip}{1em}


%\newcommand{\ac}[2]{\textcolor{cobalt}{\texttt{#1 [\textcolor{red}{#2}]}}}

\title{An Investigation into frameworks for Test Driven
Development on the iOS platform with further
analysis of Business Driven Development as it relates to test naming
conventions and their support or otherwise in the considered frameworks}

\author{Tim Storey \hfill for \hfill ICSA Software Ltd.
        \\\texttt{tim.storey@mail.com}}

\date{March, 2016}

\begin{document}


\maketitle

\begin{abstract}
  The document presents an analysis of various testing frameworks
  namely,
  \\Specta\footnote{\url{https://github.com/specta/specta}}, a testing
  framework that supports BDD type syntax,
  \\Expecta\footnote{\url{https://github.com/specta/expecta}}, a matcher
  framework 
  \\Cedar\footnote{\url{https://github.com/pivotal/cedar}}, another
  framework that supports BDD syntax and has built in matchers, and XCTest, XCode's native testing library and its
  assertions.
  \\Finally we will consider Swift, mocking implications and the
  Quick\footnote{\url{https://github.com/Quick/Quick}}
  library and its matcher library Nimble\footnote{\url{https://github.com/Quick/Nimble}}. 

  The document will consider the relative merits of each in
  turn and the merits of using matchers over the native assertions of
  XCTest. The document will further consider the  BDD syntax available with the
  libraries and its merits in terms of naming as it relates to tests as code documentation
  and ease of description of both business logic and unit test as well
  as its use in ticket creation and deliverable estimations.
  
  We will also consider the ease of integration of the 
  libraries to a CI system and the possibilities offered by Quick and
  writing tests in Swift as an aid to learning the new language.
  \end{abstract}

{\hypersetup{linkcolor=black} \tableofcontents } 
%\tableofcontents
\newpage

\section{Introduction}
  \label{sec:Intro}
  Unit testing is fairly new in regards to iOS, it was first added to
  XCode5 as a native library in the form of XCTest although it has been available
  as the SenTestingKit for OSx since XCode 2. As such it has not become such an
  integral part of a development strategy as it has in other
  platforms particularly in the world of Java based server development.

  A discussion of the merits or otherwise of TDD is beyond the scope of
  this document, here we assume that it \textbf{IS} desirable to write
  code in a test first manner and as such want to help this aim with the best
  tool kit for the job.
  
  There has also been move toward allowing for a 
  BDD\footnote{\url{https://en.wikipedia.org/wiki/Behavior-driven_development}} 
  like syntax whereby a system is
  described in a language more suited to the business needs itself, this
  language attempts to match the actual behavior and expectations of
  the system.\\For example we might define a requirement as a user
  story in the following manner capturing \textit{role},
  \textit{feature} and \textit{benefit};

  \begin{spec}{0.5}
    \ac{As a }{ROLE}\\
    \ac{I want }{FEATURE}\\
    \ac{So that }{VALUE/OUTCOME}
  \end{spec}

  Then we might decompose each part of the user story in a
  manner that explains how the user expects the system to behave, i.e. we
  would describe a scenario in terms of acceptance criteria;

  \begin{spec}{0.5}
    \ac{Given:}{some CONTEXT}\\
    \ac{When:}{some EVENT}\\
    \ac{Then:}{some ACTION}
  \end{spec}

  It is becoming common to specify behaviors this way as it allows
  scenarios to be described in manner which allows developers to build
  up code that matches the requirements easily without wasted effort and
  it also provides acceptance criteria. It is however worth noting that
  this will never be perfect and a degree of pragmatism is as always
  needed.
  
  XCTest is Apple's version of OCUnit/SenTestingKit and is quite an
  old library,
  as such whilst OCUnit works well it was never really written with more
  modern testing practices in mind, by which I mean the use of a BDD like syntax. It does
  not have any support for matchers (although these can be added) or context based tests furthermore
  asynchronous testing is only a recent addition. These 
  limitations that are fine when testing in a \textit{Test Last}
  situation (i.e. Unit Testing) and also work fine for simple
  TDD are less useful when we wish to consider our code as not only
  fulfilling requirements but also acting as documentation for later
  developers (ourselves included!)\\

\section{BDD syntax}
  BDD\footnote{\url{https://en.wikipedia.org/wiki/Behavior-driven_development}}
  is a relatively new (it was formally defined in 2009) process that is in
  essence an extension of TDD methodologies. As briefly explained in the
  section~\nameref{sec:Intro} it is a means of extending TDD and
  providing a syntax that closely matches Agile's (and to some extent UML's) User
  Stories which further creates acceptance testing criteria. 
  It has a simple syntax and
  grammar that allows the capture and expression of concepts
  understandable to the whole business.

  Essentially we can view this as an attempt (and rather a successful
  one) at bridging the languages that exists through varying levels of
  a business, from Product Owners to the Business and from Developers to
  Business Analysts/Project Manager. 
  
  To explain this a little more most commonly a business will tend to be split
  into three language \textit{levels}, by which I mean three different approaches
  to the common goal and thus three entirely different manners of
  expressing the goal.

  The \textit{levels} might be considered to be something like
  
  \begin{enumerate}
    \item{Business/Product Owner}
    \item{Business Analyst/Product Manager}
    \item{Developers}
  \end{enumerate}
  
  Using BDD we try to create a bridge between all the \textit{levels} and thus
  work toward ensuring that the team communicates each part's needs
  effectively to the other parts of the business. 

  As such we end up with a series of logical \textit{contexts} that define
  units of work, both in terms of business logic (that managers
  understand) and functional logic (that developers understand).

  There is no formal strict specification of how this should be done,
  the given team must reach its own agreement of the best means of
  describing the various levels of story.

  For example let us consider the following requirement that is actually
  in an existing ticket. 

  \subsection{A worked Example}

  The business wants an icon to be displayed on a cell if the document can be backed
  up using CrossPlatformSync, a new feature just added to the product. 
  This is a simple example but shows the general idea.

  We might capture this as a User Story in the following manner
  \begin{tspec}{1.0}{User Story}
    \centering
      \ac{As a}{USER}\\
      \ac{I want}{To see an icon displayed on a document icon}\\
      \ac{So that}{I know I can backup my document across platforms}
  \end{tspec}

  We can break this down further by driving out the \textbf{Acceptance
  Criteria}.
  
  These allow us to know when the task can be considered
  complete, these criteria are sometimes referred to as
  \textbf{scenarios} or sometimes \textbf{specifications} and this
  naming tends to be determined by the tooling used to write the tests.
  For our purposes we will refer to these as specifications or specs.
  
  So we end up with a spec that might look something like this.

  \begin{tspec}{1}{Cross platform sync icon display}
      \ac{Given}{I am working online}\\
    \ac{And}{I have logged in successfully}\\
    \ac{And}{I have documents that I have permissions to view}\\
    \ac{And}{I have downloaded these documents}\\
    \ac{Then}{I want to see those documents listed}\\
    \ac{And}{Cross platform sync is available}\\
    \ac{Then}{I want to see the cross platform sync icon}\\
  \end{tspec}

  As can be seen we have now managed to drive out a lot of separate
  acceptance criteria that are all involved in the original user story \hyperref[Ex:User Story]{Example 2.1 User Story}
  so we might think about creating some more user stories, perhaps
  something along the lines of

  \begin{itemize}
    \item{User authorization}
    \item{Document download}
  \end{itemize}

  These themselves may have acceptance criteria look something like

  \begin{tspec}{1}{User Authorization}
    \ac{As a}{USER}\\
    \ac{I want}{To be able to login}\\
    \ac{So that}{I can use the application}
  \end{tspec}

  \begin{tspec}{1}{Document download}
    \ac{As a}{USER}\\
    \ac{I want}{To be able see a list of my documents}\\
    \ac{So that}{I can use them}\\
  \end{tspec}

  All of the derived acceptance criteria can be considered to be discrete
  logic areas that will require development work, and as such become 
  our \textbf{requirements} or in terms of a testing framework that supports BDD 
  \textbf{contexts}.
  
  We will be developing the code to reach these criteria in a test
  first manner and we can build up the tests and thus the code wrapped
  in one of the contexts that we have just defined.

  These requirements further describe the integration tests
  that QA will need to do, but we will come back to this in \hyperref[sec:integration tests]{Functional/Integration tests}
  
  This example is somewhat contrived (though based on a ticket) and could 
  be worked further but it is not the intention to
  fully explore BDD here merely to show its utility.

\section{Specs as tickets and estimation}
  As we can see the BDD syntax drives out the criteria we need to meet
  for the story to be accepted and as such we are now in a better
  position to divide the needed development work up into distinct units which can be
  captured as tickets.

  Considering the requirements for \textbf{Cross Platform Sync icon display}
  and assuming we have now refined the specs to the following;

  \begin{tspec}{1}{Display Cross Platform Sync Icon}
    \ac{Given}{There are documents displayed for a meeting}\\
    \ac{And}{Cross platform sync is enabled}\\
    \ac{Then}{I want to see the cross platform sync icon displayed}
  \end{tspec}

  the specs so far are descriptive enough to capture
  the story at a fairly high level and can be easily enough
  understood across the levels of the business  to reach agreement
  between the teams but in terms of
  development we are going to need to go deeper and create finer grained
  technical specifications that acheive the goals set out in the
  requirements, really it's these technical specs
  (that we might consider a sub category of a BDD User Story) that we
  are most interested in for the purposes of this document.

  These technical specs will define the functional requirements of the code we are
  developing.

  \subsection{Developing functional requirements and tickets}
   Let's assume that most of the code for the spec is already covered i.e. documents
  are downloaded and displayed correctly we just need to write some code
  to set a \texttt{Boolean} flag that indicates that the icon needs display.

  We have prevously decided architecture and know that the document
  icons are displayed as cells in a table view, this has a working data
  source etc so that takes care of the \texttt{Given} part of the the
  spec. 

  Given the above asumption we could now make 2 tickets to capture the
  deliverables that represent the story, if we had no data source nor
  table view then that of course would become another deliverable
  albeit a big one.
  
  So then these tickets are going to be 

  \begin{enumerate}
    \item{Determine if cross platform sync is enabled}
    \item{Display icon if cross platform is enabled}
  \end{enumerate}

  Looking into the first one we discover that there are a couple of
  conditions that determine whether or not cross platform sync is
  enabled. These conditions should be captured somewhere, probably in the
  ticket or in someone's head but unless it's really complex that's
  probably enough for the ticket and the BDD like description of
  requirements to cover a scenario, the information we now need as
  developers is not the information that business needs to know, however
  we do now have small deliverable tickets and furthermore these tickets
  clearly capture the requirements.
  
  
  Looking at the first ticket we decide that it makes sense to create a
  small reusable, extensible helper class that can capture the logic
  we need and return the \texttt{Boolean} value that determines wether or not to
  display the icon. 
  
  The class responsible for displaying the icon can
  then use this new helper class, and we will pass it to the class as a
  dependency thus decoupling our logic like good nerds.

  We will build this code up under test and we are therefore likely
  going to need two test cases one for the new class and one for the
  class that has a dependency on this new helper class.

  We know that we determine the availability of CrossPlatformSync based
  on server version and on permissions held in the users organisation
  profile.

  The server version is available to the table view that configures the
  cell and the users permissions regarding backup are held in the
  organisation profile.

  Therefore we decide to make a simple helper class that exposes a boolean
  function to which we pass the server version and the organisation
  profile.

  Assuming we are calling the class CrossPlatformSyncHelper we might
  end up with a test class somthing like the following (for the sake of
  brevity a lot of boilerplate has been removed);

  \begin{listbox}{1}{XCTest Case first parse}
      - (void)test_InitializationCalls
      {
          XCTAssertNotNil(self.crossPlatformSyncSupportHelperUnderTest, @"Did not init");
      }

      - (void)test_initialisesWithAMinimumServerVersion
      {
          self.crossPlatformSyncSupportHelperUnderTest = [CrossPlatformSyncSupportHelper supportHelperWithMinimumDocumentServerVersion:1];
          XCTAssertNotNil(self.crossPlatformSyncSupportHelperUnderTest, @"Failed to initialise class");
          XCTAssertEqual(self.crossPlatformSyncSupportHelperUnderTest.minimumDocumentServerVersion, 1, @"Failed to set server version");
      }

      - (void)test_givenADocumentServerVersionAboveZeroAndAnOrganisationProfileThatAllowsBackupThenCrossPlatformSyncReturnsTrue
      {
          [self mockBoardData];
          OrganisationProfile *organisationProfile = [OrganisationProfile insertInManagedObjectContext:self.organisationProfileManagedObjectContext];
          organisationProfile.AllowBackup = [NSNumber numberWithBool:YES];

          OCMStub([self.mockedBoardData currentOrganisationProfile]).andReturn(organisationProfile);
          XCTAssertTrue([_crossPlatformSyncSupportHelperUnderTest
          isCrossPlatformSyncEnabledForDocumentServerVersion:1
          withProfile: [self.mockedBoardData currentOrganisationProfile]], @"Does not return true for server version 1");
      }

      - (void)test_givenAnOrgProfileThatHasSyncEnabledLocalTestForSyncEnabledReturnsTrue
      {
          [self mockBoardData];
          OrganisationProfile *organisationProfile = [OrganisationProfile insertInManagedObjectContext:self.organisationProfileManagedObjectContext];
          organisationProfile.AllowBackup = [NSNumber numberWithBool:YES];

          XCTAssertTrue([_crossPlatformSyncSupportHelperUnderTest isCrossPlatformSyncEnabledForOrganisationProfile:organisationProfile], @"Does not return TRUE when valid in org profile");
      }
  \end{listbox}


  So we can now see how using BDD we can start with a high level
  \textit{User
  Story} or \textit{Main Storey} or \textit{Use Case}, pick your own preferred term here
  that the Business and Preoject Manager and Business Analyst can
  understand and agree upon. 
  
  This then can be used to drive out further requirements, which in turn
  creates some acceptance criteria which are in turn deliverables or
  tickets and so forth and possibly most importantly each ``level'' of the
  business can now describe their own requirements in a manner that
  suits them.

  Hopefully it can also be seen that this leads to the posiiblity of
  testing as documention but we will will get back to this.
\section{Specta test framework}
  Specta is a well supported and well used testing framework available
  as source and as library and hosted on Github.

  It is an open source project and well supported and used (1647 stars
  at last count) and still under development, its is hard to find a list
  of companies suing it but the 
\section{Cedar framework}
\section{Quick framework and the Nimble matcher framework}
\section{Matchers vs Assertions}
\section{Test naming as documentation}
\section{Naming in relation to tickets (BDD again)}
\section{Integration with XCode}
\section{Support implications}
  \section{Swift and possibilities for learning via testing}
\section{Pros and cons}
\section{Integration to CI}
\section{Functional/Integration tests}\label{sec:integration tests}
  Integration testing with Kalabash and Cucumber
\section{Conclusion}
  
\end{document}

%@implementation CrossPlatformSyncSupportHelper
%
%- (BoardData*)boardData
%{
%    if (_boardData == Nil)
%    {
%       _boardData = [BoardData boardData];
%    }
%
%    return _boardData;
%}
%
%+ (CrossPlatformSyncSupportHelper *)supportHelperWithMinimumDocumentServerVersion:(int)minimumServerVersion
%{
%    return [[CrossPlatformSyncSupportHelper alloc] initWithMinimumDocumentServerVersion:minimumServerVersion];
%}
%
%- (instancetype)initWithMinimumDocumentServerVersion:(NSInteger)minimumServerVersion
%{
%    self = [super init];
%    if (self)
%    {
%        _minimumDocumentServerVersion = minimumServerVersion;
%    }
%
%    return self;
%}
%
%
%- (BOOL)isCrossPlatformSyncEnabledForDocumentServerVersion:(NSInteger)serverVersion
%{
%    if ([self isCrossPlatformSyncEnabledForOrganisationProfile:[self.boardData currentOrganisationProfile]])
%    {
%        if ([self isCorrectServerVersion:serverVersion])
%        {
%            return YES;
%        }
%        else
%        {
%            return NO;
%        }
%    }
%    else
%    {
%        return NO;
%    }
%}
%
%- (BOOL)isCorrectServerVersion:(NSInteger)serverVersion
%{
%    if (serverVersion >= _minimumDocumentServerVersion)
%    {
%        return YES;
%    }
%    else
%    {
%        return NO;
%    }
%}
%
%
%- (BOOL)isCrossPlatformSyncEnabledForOrganisationProfile:(OrganisationProfile *)organisationProfile
%{
%    if ([organisationProfile.AllowBackup boolValue] == YES )
%    {
%        return YES;
%    }
%    else
%    {
%        return NO;
%    }
%}
%@end


%- (void)setUp
%{
%    [super setUp];
%    self.crossPlatformSyncSupportHelperUnderTest = [CrossPlatformSyncSupportHelper supportHelperWithMinimumDocumentServerVersion:1];
%}
%
%- (void)tearDown
%{
%    self.crossPlatformSyncSupportHelperUnderTest = nil;
%    [super tearDown];
%}
%
%- (void)test_InitializationCalls
%{
%    XCTAssertNotNil(self.crossPlatformSyncSupportHelperUnderTest, @"Did not init");
%}
%
%- (void)test_initialisesWithAMinimumServerVersion
%{
%    self.crossPlatformSyncSupportHelperUnderTest = [CrossPlatformSyncSupportHelper supportHelperWithMinimumDocumentServerVersion:1];
%    XCTAssertNotNil(self.crossPlatformSyncSupportHelperUnderTest, @"Failed to initialise class");
%    XCTAssertEqual(self.crossPlatformSyncSupportHelperUnderTest.minimumDocumentServerVersion, 1, @"Failed to set server version");
%}
%
%- (void)test_afterInitialisationWeHaveAMockedBoardDataInstanceToUse
%{
%    [self mockBoardData];
%    _crossPlatformSyncSupportHelperUnderTest.boardData = self.mockedBoardData;
%    XCTAssertNotNil(self.crossPlatformSyncSupportHelperUnderTest.boardData, @"Did not set Blueprint property");
%    XCTAssertEqualObjects(self.crossPlatformSyncSupportHelperUnderTest.boardData, self.mockedBoardData, @"We are not using a mocked blueprint instance");
%}
%
%- (void)test_givenADocumentServerVersionAboveZeroAndAnOrganisationProfileThatAllowsBackupThenCrossPlatformSyncReturnsTrue
%{
%    [self mockBoardData];
%    OrganisationProfile *organisationProfile = [OrganisationProfile insertInManagedObjectContext:self.organisationProfileManagedObjectContext];
%    organisationProfile.AllowBackup = [NSNumber numberWithBool:YES];
%
%    OCMStub([self.mockedBoardData currentOrganisationProfile]).andReturn(organisationProfile);
%    XCTAssertTrue([_crossPlatformSyncSupportHelperUnderTest isCrossPlatformSyncEnabledForDocumentServerVersion:1], @"Does not return true for server version 1");
%}
%
%- (void)test_givenAnOrgProfileThatHasSyncEnabledLocalTestForSyncEnabledReturnsTrue
%{
%    [self mockBoardData];
%    OrganisationProfile *organisationProfile = [OrganisationProfile insertInManagedObjectContext:self.organisationProfileManagedObjectContext];
%    organisationProfile.AllowBackup = [NSNumber numberWithBool:YES];
%
%    XCTAssertTrue([_crossPlatformSyncSupportHelperUnderTest isCrossPlatformSyncEnabledForOrganisationProfile:organisationProfile], @"Does not return TRUE when valid in org profile");
%}
